%%%%%%%%%%%%%%%%%%%%%%%%%%%%%%%%%%%%%%%
% Deedy - One Page Two Column Resume
% LaTeX Template
% Version 1.1 (30/4/2014)
%
% Original author:
% Debarghya Das (http://debarghyadas.com)
%
% Original repository:
% https://github.com/deedydas/Deedy-Resume
%
% IMPORTANT: THIS TEMPLATE NEEDS TO BE COMPILED WITH XeLaTeX
%
% This template uses several fonts not included with Windows/Linux by
% default. If you get compilation errors saying a font is missing, find the line
% on which the font is used and either change it to a font included with your
% operating system or comment the line out to use the default font.
% 
%%%%%%%%%%%%%%%%%%%%%%%%%%%%%%%%%%%%%%
% 
% TODO:
% 1. Integrate biber/bibtex for article citation under publications.
% 2. Figure out a smoother way for the document to flow onto the next page.
% 3. Add styling information for a "Projects/Hacks" section.
% 4. Add location/address information
% 5. Merge OpenFont and MacFonts as a single sty with options.
% 
%%%%%%%%%%%%%%%%%%%%%%%%%%%%%%%%%%%%%%
%
% CHANGELOG:
% v1.1:
% 1. Fixed several compilation bugs with \renewcommand
% 2. Got Open-source fonts (Windows/Linux support)
% 3. Added Last Updated
% 4. Move Title styling into .sty
% 5. Commented .sty file.
%
%%%%%%%%%%%%%%%%%%%%%%%%%%%%%%%%%%%%%%%
%
% Known Issues:
% 1. Overflows onto second page if any column's contents are more than the
% vertical limit
% 2. Hacky space on the first bullet point on the second column.
%
%%%%%%%%%%%%%%%%%%%%%%%%%%%%%%%%%%%%%%

\documentclass[]{deedy-resume-openfont}


\begin{document}


%%%%%%%%%%%%%%%%%%%%%%%%%%%%%%%%%%%%%%
%
%     TITLE NAME
%
%%%%%%%%%%%%%%%%%%%%%%%%%%%%%%%%%%%%%%


\namesection{Radek}{Starosta}{Španielova 1313, Praha 6 - Řepy, 163 00 \\
    \href{mailto:radek.starosta@gmail.com}{radek.starosta@gmail.com} | 720 365 201 \\
}

\begin{minipage}[t]{0.8\textwidth} 

\sectionsep

%%%%%%%%%%%%%%%%%%%%%%%%%%%%%%%%%%%%%%
%     EDUCATION
%%%%%%%%%%%%%%%%%%%%%%%%%%%%%%%%%%%%%%

\section{Vzdělání} 


\runsubsection{ČVUT FEL OI}
\descript{| Softwarové systémy}
\location{2014 - ? | Praha, ČR}
\sectionsep

\runsubsection{Gymnázium Arabská}
\descript{| Programování}
\location{2010 - 2014 | Praha, ČR}
\vspace{\topsep}
\begin{tightemize}
\item V průběhu studia jsem pracoval na několika ročníkových projektech:
    \begin{itemize}
    \item formátování a zvýrazňování zdrojových kódů (Java)
    \item program pro tvorbu školních rozvrhů (Java)
    \item 3D hra s tratí generovanou podle zvukové stopy (C\#)
    \end{itemize}
\item Podílel jsem se na správě školního studentského serveru a spoluautorem webové aplikace pro jeho obsluhu.
\end{tightemize}
\sectionsep

\runsubsection{ZŠ Jana Wericha}
\descript{}
\location{2001 - 2010 |  Praha, ČR}
\sectionsep

%%%%%%%%%%%%%%%%%%%%%%%%%%%%%%%%%%%%%%
%     EXPERIENCE
%%%%%%%%%%%%%%%%%%%%%%%%%%%%%%%%%%%%%%

\section{Pracovní zkušenosti}

\runsubsection{ČVUT FEL}
\descript{| Letní brigáda CMP }
\location{červenec - srpen 2013 | Praha, ČR}
\begin{tightemize}
\item Plugin pro rozpoznávání tváří na fotkách do prohlížeče Google Chrome (JavaScript)
\item Galerie zvýrazňující anotované prvky obličeje. (PHP)
\end{tightemize}
\sectionsep


%%%%%%%%%%%%%%%%%%%%%%%%%%%%%%%%%%%%%%
%     SKILLS
%%%%%%%%%%%%%%%%%%%%%%%%%%%%%%%%%%%%%%

\section{Schopnosti}
\subsection{Programování}
\location{Pokročilý:}
Java \textbullet{} PHP \\
\location{Mírně pokročilý:}
C\# \textbullet{} JavaScript \textbullet{} SQL \textbullet{} \LaTeX \\
\location{Základní:}
C \textbullet{} Shell \textbullet{} Ruby \textbullet{} CSS
\sectionsep

\subsection{Jazyky}
\location{Pokročilý:}
Anglický (\textasciitilde  C1) \\
\location{Základní:}
Německý
\sectionsep

\end{minipage}
\end{document}  \documentclass[]{article}
